\section{Introduction}
Job scheduling is a fundamental and important problem in computer science and operations research, and over the last sixty years, the relevance of scheduling has only increased. In today's world
of data centers and cloud computing, countless scheduling tasks are performed each second which are executed over various inter and intra networks. These developments (cite) underscore the importance of
communication costs when deciding how and where to schedule jobs. The traditional viewpoint that jobs can reach a machine instantaneously is often compromised, and one has to look at {\em network scheduling} (cite)
as the correct paradigm.  In this model, broadly speaking, machines and jobs exist in an underlying network and the schedule must take into account communication costs while optimizing the objective function.
\smallskip

In most network scheduling problems studied so far, one assumes the positions of jobs and machines to be fixed as input to the problem. In a typical data center application, since the jobs are generated by users it is fair to assume no control on where these
jobs originate. The latter assumption of the positions of machines being fixed however is questionable, since in the network-planning phase the application indeed has a choice in which locations it can place its machines. We argue that the machine locations should
also be {\em part of the optimization problem} rather than being fixed arbitrarily (or worse, adversarially). This is the angle we raise in this paper, and we address one question in this setting.
\smallskip

Formally, we are given a collection of jobs $J$ which lie on a symmetric metric space $(X,d)$ where $d(i,j)$ indicated the time taken to send a job from location $i$ to location $j$.
%In this paper we {\em ignore congestion issues} in communication/transportation of jobs from one location to another.
Each job has a processing time $p_j$ and a location $j \in X$ (abusing notation for brevity's sake). Furthermore, we have at our disposal a collection of machines of {\em differing speeds} -- we have $n_1$ machines of speed $s_1$, $n_2$ machines of speed $s_2$,
and so on up to $n_t$ machines of speed $s_t$. As usual, job $j$ can be processed of a machine $i$ in time $p_j/s_i$.
The goal is to decide the locations of these machines on the metric space {\em and} figure out a schedule of the jobs to machines; in this paper we look at the classic objective of minimizing makespan, the time taken for a job
to reach the machine it is assigned to and complete processing on that machine.

\begin{itemize}
	\item Connection to capacitated k-center (barilan, khuller)
	\item A new capacitated k-center problem
	\item Result
\end{itemize}
\paragraph{Capacitated $k$-Center Problem with Movable Capacities (\mckc).} In this problem, we are given as input a metric space $(X,d)$, and a collection of capacities: $(n_1,c_1), (n_2,c_2),\ldots, (n_t,c_t)$
with $c_1 \leq c_2 \le \cdots \le c_t$,
to indicate we have $n_i$ centers of capacity $c_i$ (these are called type $i$ centers) which we can open in $X$. The goal is to find locations to open these centers and assign all points of $X$ to one of these so that (a) any center of type $i$ serves at most $c_i$ locations, and (b) the maximum distance of a location $j$ to its assigned center $i$ is minimized. We use $\opt$ to denote this latter distance of the optimum solution.


\begin{definition}
An $(a,b)$-bicriteria algorithm for \mckc can assign locations to a center at most $a\cdot\opt$ away, and assigns each center of type $i$ at most $b\cdot c_i$ locations. In other words, the capacity of each type $i$ center is $b \cdot c_i$.
\end{definition}

